\documentclass[a4paper,12pt]{article}
\usepackage[left=3cm, right=2cm, top=2.5cm, bottom=2.5cm]{geometry} % Căn lề chuẩn A4
\usepackage{fontspec}
\usepackage{polyglossia}
\usepackage{setspace} % Điều chỉnh khoảng cách dòng
\usepackage{fancyhdr} % Quản lý header/footer
\usepackage{graphicx} % Hỗ trợ chèn ảnh
\usepackage{tocloft} % Chỉnh sửa mục lục
\usepackage{hyperref} % Tạo liên kết mục lục
\usepackage{xcolor} % Định dạng màu sắc liên kết
\usepackage{lipsum} % Nội dung giả (có thể xóa khi hoàn thành)

\setmainlanguage{russian}
\setmainfont{Times New Roman} % Nếu lỗi, thử "Liberation Serif" hoặc "FreeSerif"
\newfontfamily\russianfont{Times New Roman}

% Cấu hình header/footer và đánh số trang từ trang 2
\pagestyle{fancy} 
\fancyhf{} % Xóa header/footer mặc định
\fancyfoot[C]{\thepage} % Số trang ở giữa footer
\renewcommand{\headrulewidth}{0pt} % Không có gạch dưới header
\renewcommand{\footrulewidth}{0pt} % Không có gạch trên footer

% Cấu hình số thứ tự tiêu đề bắt đầu từ "Ход работы"
\setcounter{secnumdepth}{2} % Đánh số tới subsection
\setcounter{tocdepth}{2} % Hiển thị đến subsection trong mục lục

% Thiết lập màu sắc cho liên kết mục lục
\hypersetup{
	colorlinks=true,
	linkcolor=blue, % Liên kết có màu xanh
	urlcolor=blue
}

\begin{document}
	
	% TRANG TIÊU ĐỀ (không có số trang)
	\thispagestyle{empty} % Trang đầu không có số trang
	
	\onehalfspacing % Giãn dòng 1.5 để dễ đọc
	
	\begin{center}
		\textbf{Министерство науки и высшего образования Российской Федерации} \\
		\textbf{ФЕДЕРАЛЬНОЕ ГОСУДАРСТВЕННОЕ АВТОНОМНОЕ ОБРАЗОВАТЕЛЬНОЕ} \\
		\textbf{УЧРЕЖДЕНИЕ ВЫСШЕГО ОБРАЗОВАНИЯ} \\
		\textbf{НАЦИОНАЛЬНЫЙ ИССЛЕДОВАТЕЛЬСКИЙ УНИВЕРСИТЕТ ИТМО} \\
		\vspace{1cm}
		\textbf{Факультет безопасности информационных технологий} \\
		\vspace{0.5cm}
		\textbf{Дисциплина:} \\
		«Основы теории надежности» \\
		\vspace{1cm}
		\textbf{ОТЧЕТ ПО ЛАБОРАТОРНОЙ РАБОТЕ №2} \\
		«Анализ дерева неисправностей (FTA)» \\
	\end{center}
	
	\vspace{1.5cm} % Điều chỉnh khoảng cách xuống dưới
	
	\begin{flushright}
		\textbf{Выполнил:} \\
		Чу Ван Доан, студент группы N3347 \\
		
		\vspace{0.25cm} % Khoảng cách trước khi chèn ảnh
		\includegraphics[width=3cm]{chuky.png} % Đường dẫn ảnh chữ ký (thay "chuky.png" bằng tên ảnh của bạn)
		
		\rule{5cm}{0.4pt} \\ (подпись)
	\end{flushright}
	
	\begin{flushright}
		\textbf{Проверил:} \\
		Мухамеджанов Санжар \\
		
		\vspace{1cm}
		\rule{5cm}{0.4pt} \\ (отметка о выполнении) \\
		\vspace{0.5cm}
		\rule{5cm}{0.4pt} \\ (подпись)
	\end{flushright}
	
	\vfill
	
	\begin{center}
		Санкт-Петербург \\
		2025 г.
	\end{center}
	
	\newpage
	
	% MỤC LỤC
	\section*{\label{sec:toc_main}СОДЕРЖАНИЕ} % Đặt nhãn liên kết
	\addcontentsline{toc}{section}{\hyperref[sec:toc_main]{СОДЕРЖАНИЕ}} % Thêm "СОДЕРЖАНИЕ" vào mục lục với liên kết
	
	% Hiển thị mục lục
	\pagenumbering{arabic} % Đánh số trang dạng 1, 2, 3...
	\setcounter{page}{2} % Trang tiếp theo là số 2
	
	\tableofcontents 
	\newpage
	
	% CÁC PHẦN TRONG BÁO CÁO
	\section*{Введение} % Không đánh số
	\addcontentsline{toc}{section}{Введение} % Thêm vào mục lục nhưng không có số thứ tự
	\lipsum[1] % Nội dung giả, thay thế bằng nội dung thực tế
	
	\section*{Задание} % Không đánh số
	\addcontentsline{toc}{section}{Задание} % Thêm vào mục lục nhưng không có số thứ tự
	\lipsum[2]
	
	% Bắt đầu đánh số tiêu đề từ đây
	\section{Ход работы}
	\subsection{Атака с использованием фишинга}
	\subsubsection{Вид угрозы}
	\lipsum[3]
	
	\subsubsection{Источник угрозы}
	\lipsum[4]
	
	\subsubsection{Способ реализации}
	\lipsum[5]
	
	\subsubsection{Цель}
	\lipsum[6]
	
	\subsubsection{Меры предосторожности}
	\paragraph{Технические меры}
	\lipsum[7]
	
	\paragraph{Организационные меры}
	\lipsum[8]
	
	\section{DDoS-атака на корпоративный веб-сайт}
	\subsection{Вид угрозы}
	\lipsum[9]
	
	\subsection{Источник угрозы}
	\lipsum[10]
	
	\subsection{Способ реализации}
	\lipsum[11]
	
	\subsection{Цель}
	\lipsum[12]
	
	\subsection{Меры предосторожности}
	\paragraph{Технические меры}
	\lipsum[13]
	
	\newpage % Xuống trang mới, bắt đầu từ đây sẽ có số trang
	
\end{document}
